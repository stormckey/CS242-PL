
\documentclass[11pt]{article}

\input{../../tex/defs.tex}

% Useful syntax commands
\newcommand{\jarr}[1]{\left[#1\right]}   % \jarr{x: y} = {x: y}
\newcommand{\jobj}[1]{\left\{#1\right\}} % \jobj{1, 2} = [1, 2]
\newcommand{\pgt}[1]{\, > {#1}}          % \pgt{1} = > 1
\newcommand{\plt}[1]{\, < {#1}}          % \plt{2} = < 2
\newcommand{\peq}[1]{\, = {#1}}          % \peq{3} = = 3
\newcommand{\prop}[1]{\langle{#1}\rangle}% \prop{x} = <x>
\newcommand{\matches}[2]{{#1}\sim{#2}}   % \matches{a}{b} = a ~ b
\newcommand{\aeps}{\varepsilon}          % \apes = epsilon
\newcommand{\akey}[2]{.{#1}\,{#2}}       % \akey{s}{a} = .s a
\newcommand{\aidx}[2]{[#1]\,{#2}}        % \aidx{i}{a} = [i] a
\newcommand{\apipe}[1]{\mid {#1}}        % \apipe{a} = | a

% Other useful syntax commands:
%
% \msf{x} = x (not italicised)
% \falset = false
% \truet = true
% \tnum = num
% \tbool = bool
% \tstr = str


\begin{document}

\hwtitle
  {Assignment 1}
  {Stormckey} %% REPLACE THIS WITH YOUR NAME/ID

\problem{Problem 1}

Part 1:

\begin{alignat*}{1}
\msf{Property}~p ::= \qamp \varepsilon \\
\mid \qamp np \\
\mid \qamp sp \\
\\
\msf{Numerical\ Property}~np ::= \qamp = n \\
\mid \qamp > n \\
\mid \qamp < n \\
\mid \qamp np \wedge np \\
\mid \qamp np \vee np \\
\mid \qamp (np) \\
\\
\msf{String\ Property}~sp ::= \qamp = s \\
\mid \qamp sp \vee sp \\
\\
\msf{Schema}~\tau ::= \qamp \tnum \prop{np} \\
\mid \qamp \tnum \prop{\varepsilon} \\
\mid \qamp \tstr \prop{sp} \\
\mid \qamp \tstr \prop{\varepsilon} \\
\mid \qamp \tbool \\
\mid \qamp [\tau^*] \\
\mid \qamp \{(s:\tau)^*\} \\
\end{alignat*}

Part 2:

% mathpar is the environment for writing inference rules. It takes care of
% the spacing and line breaks automatically. Use "\\" in the premises to
% space out multiple assumptions.
\begin{mathpar}

\ir{S-Bool-False}{\ }{\matches{\falset}{\tbool}}

\ir{S-Bool-True}{\ }{\matches{\truet}{\tbool}}

\ir{S-Num}{\ }{\matches{n}{\tnum \prop{\varepsilon}}}

\ir{S-Num-Equal}{n1 = n2 }{\matches{n_1}{\tnum \prop{= n_2}}}

\ir{S-Num-Less}{n_1 < n_2}{\matches{n_1}{\tnum \prop{< n_2}}}

\ir{S-Num-Greater}{n_1 > n_2}{\matches{n_1}{\tnum \prop{> n_2}}}

\ir{S-Num-And}{\matches{n}{\tnum \prop{p}} , \matches{n}{\tnum \prop{q}}}{\matches{n}{\tnum \prop{p \wedge q}}}

\ir{S-Num-Or}{\matches{n}{\tnum \prop{p}} \vee \matches{n}{\tnum \prop{q}}}{\matches{n}{\tnum \prop{p \vee q}}}

\ir{S-Num-Brackets}{\matches{n}{\tnum \prop{p}}}{\matches{n}{\tnum \prop{(p)}}}

\ir{S-Str}{\ }{\matches{s}{\tstr \prop{\varepsilon}}}

\ir{S-Str-Equal}{s_1 = s_2}{\matches{s_1}{\tstr \prop{= s_2}}}

\ir{S-Str-Or}{\matches{s}{\tstr \prop{p}} \vee \matches{s}{\tstr \prop{q}}}{\matches{s}{\tstr \prop{p \vee q}}}

\ir{S-Array}{\forall i = 0 \ldots |j| -1. \matches{a_i}{\tau_i}}{\matches{[a^*]}{[\tau^*]}}

\ir{S-Dict}{\forall s' \in s. \matches{j_{s'}}{\tau_{s'}}}{\matches{\{(s : j)^*\}}{\{(s : \tau)^*\}}}

% Inference rules go here
\end{mathpar}


\problem{Problem 2}

Part 1:

\begin{mathpar}
% Inference rules go here
\ir{E-Access}{s' \in s}{\steps{(.s'a,\{(s,j)^*\})}{(a,j_{s'})}}

\ir{E-Index}{i \in \{0 \ldots |j| - 1\}}{\steps{([i]a,[j^*])}{(a,j_i)}}

\ir{E-Map-Access}{\forall i= 1 \ldots |[\{s,j\}^*]|-1, s'\in s_i}{\steps{(|.s'a,[\{s,j\}^*])}{[j_{i,s'}]}}

\ir{E-Map-Index}{\forall l = 1 \ldots |[[j^*]^*]|-1,i \in {0,\ldots,|j_l|-1}}{\steps{(|[i]a,[[j^*]^*])}{[j_{l,i}^*]}}
\end{mathpar}

Part 2:

\begin{mathpar}
% Inference rules go here
\ir{V-Non}{\ }{\matches{\varepsilon}{\tau}} 

\ir{V-Access}{s' \in s, \matches{a}{\tau_{j_{s'}}}}{\matches{.s' a}{\{(s,j)^*\}}}

\ir{V-Index}{i \in \{0 \ldots |j| - 1\}, \matches{a}{\tau_{j_i}}}{\matches{[i]a}{[j^*]}}

\ir{V-Map-Access}{\forall i= 1 \ldots |[\{s,j\}^*]|-1, s'\in s_i, \matches{a}{\tau_{j_{i,s'}}}}{\matches{.|s'a}{[\{s,j\}^*]}}

\ir{V-Map-Index}{\forall l = 1 \ldots |[[j^*]^*]|-1,i \in {0,\ldots,|j_l|-1}, \matches{a}{\tau_{j_{l,i}}}}{\matches{|[i]a}{[[j^*]^*]}}
\end{mathpar}

\textit{Accessor safety}: for all $a, j, \tau$, if $\matches{a}{\tau}$ and $\matches{j}{\tau}$, then there exists a $j'$ such that $\evals{(a, j)}{\aeps, j'}$.

\begin{proof}
% Proof goes here.
Induction on the steps of derivation:

(1 step derivation)

trivial by case analysis of a.

(k step derivation, k > 1)

take 1 step like we did above then use the k-1 hypothesis.


\end{proof}

\end{document}
